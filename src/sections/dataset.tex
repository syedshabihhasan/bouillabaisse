\section{Dataset}
\begin{table}
\centering
%Overall
%Twitter 647
%FB 21190
%FB(Other) 23098
%SMS 42551
\caption{Dataset Composition by Message Type}
\begin{tabular}{| c | c |}
\hline
\textbf{Data Type} & \textbf{Count} \\
\hline
SMS & 42551\\
\hline
Facebook & 21190\\
\hline
Facebook(Other) & 23098\\
\hline
Twitter & 647\\
\hline
\end{tabular}
\label{table:datasetCompositionMsgT}
\end{table}
As mentioned in the previous section, we collect data from several sources. 
In particular, Facebook offered the widest variety \emph{viz.} messages, likes, status updates, and comments. 
We separate this in two groups, one containing the messages(Facebook), the other containing the rest of the Facebook data( Facebook(Other)). 
Similarly for Twitter, we created a single group containing the Twitter messages, and the status updates. 
The distribution of the collected data is shown in Table \ref{table:datasetCompositionMsgT}. 
The largest fraction came from the text messaging (SMS), followed by Facebook, and finally Twitter. 
For this work we only focus on the text messaging, and the Facebook message data. 

